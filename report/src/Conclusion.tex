\section{Conclusion}

We have shown that both cyclostationary detection and fourth-order cumulant
statistics can be used as a means of modulation classification given idealized
conditions. Both methods were tested in varying levels of SNR, using simulation
data in Octave and were able to identify differences between AM, Single Side Band AM, FM, BPSK, QAM, 16QAM, and 64QAM. 
When the RTL hardware receiver was used, the cumulant method was found to be
phase invariant but still required precise frequency synchronization due to its
use of time domain statistics.  In order for the cumulant method to prove
useful in a real life application, a frequency synchronization block must be
implemented and placed prior to the cumulant classifier.  The cyclostationary
approach used frequency domain statistics and did not need frequency or phase
synchronization but was unable to identify modulation characteristics using
the RTL hardware samples without using prior knowledge of the signals spectrum
as a template.  The real life spectral shapes different from those of the
simulation due to unknown modulation parameters such as roll off factors,
frequency division, and  matched filtering.
