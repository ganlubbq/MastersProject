\chapter{Low Budget Software Defined Radios}

The Software Defined Radio (SDR) hobby community has generally rallied around
the Universal Software Radio Peripheral (USRP) however the costs of a USRP can
range from \$1500 to \$3000 depending on daughter board configuration and
application.  This has prevented wide adotption among the more novice or budget
focued developer. In early 2012 a linux kernal developer, Antti Palosaari, found
that the RTL2832U chipset found within a low cost USB DVB-T tuner
could be used as a simple SDR.  Due to their low cost, around \$12 to \$20, a
small community has cropped up and begun development of GnuRadio blocks and
applications.

The RTL2832U chipset is found in a variety of tuner models and provide a range of frequencies.
Table~\ref{tab:ModelAndFreq} shows the supported hardware models and cooresponding frequency ranges.

\begin{table}
\caption{Tuner Model and Frequency Range Supported}
\begin{tabular}{| l | l |} \hline
Tuner	& Frequency range \\ \hline
Elonics E4000	& 52 - 1100 Mhz and 1250 - 2200 MHz 1250 MHz \\ \hline
Rafael Micro R820T &	24 - 1766 MHz \\ \hline
Fitipower FC0013 & 22 - 1100 MHz  \\ \hline
Fitipower FC0012 &	22 - 948.6 MHz \\ \hline
FCI FC2580	& 146 - 308 MHz and 438 - 924 MHz \\ \hline
\end{tabular}
\label{tab:ModelAndFreq}
\end{table}


 - Low cost SDR
 	- Discuss other options
 	- USRP (Too expensive)
 	- (Google some of the other options)
 	- Settle on the RTL-SDR, discuss properties
 	- Discuss existing GNU-Radio blocks
 
 - Practical application
	- Long term goal to make a distributed network of cheap SDRs
	- Network of signal detction and classification SDRs.
	- Use within gnuRadio
	- Use within REDHAWK SDR
   
 - Methods which work in high SNR
 	- Discuss previous work for signal detection here (From pattern recog paper)
 	- Introduce Cyclostationarity and cite the related papers with resuls for high
 	SNR
 
 - Introduction to Cyclostationary Detection
 	- Introduce basic concept
 	- Show the calculations to obtain the SCD
 	- Introduce the spectral coherence.
 	- Introduce the parameter Ia as a classification measure
 
 - Simulated Experimental results
 	- Discuss how results vary based on sample rate.
 	- Show results for different sample rates
 	- Show results vary for real and complex samples
 	- Show results for known bandwidths and noise figures
 	
 - Implementation in GnuRadio
 	-  Each detector as a block
 	- 

 - Implementation in REDHAWK
 
 - Results with snapshot and live data
 
 - Conclusions

\section{Introduction to Cyclostationary}

Much work has been done on the benefits of using Cyclostationary detection as a
means of energy detection and as a signal classification technique. 
\cite{costa96} and \cite{kim2007} do a good job of preesnting the concepts of
Cyclostationary for anyone unfarmiliar with the technique.  
\section{Literary Review of Methods}
\section{Experimental Matlab Results}
\section{RTL SDR}
\section{gnuRadio}
