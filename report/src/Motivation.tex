\chapter{Motivation}

The Software Defined Radio (SDR) hobby community has generally rallied around
the Universal Software Radio Peripheral (USRP) however the costs of a USRP can
range from \$1500 to \$3000 depending on daughter board configuration and
application.  This has prevented wide adoption among the more novice or budget
focused developer. In early 2012 a Linux kernel developer, Antti Palosaari, found
that the RTL2832U chipset found within a low cost USB DVB-T tuner
could be used as a simple SDR.  Due to their low cost, \$12 to \$20, a
small community has cropped up and begun development of GnuRadio blocks, REDHAWK
devices and applications using this hardware.

The RTL2832U chipset is found in a variety of tuner models and provide a range
of tunable frequencies.
Table~\ref{tab:ModelAndFreq} shows the supported hardware models and
corresponding frequency ranges.  The RTL2832U (RTL-SDR) chipset outputs 8-bit
I/Q samples and can support sample rates as high as 2.4 MS/s without dropping samples.  

\begin{table}
\caption{Tuner Model and Frequency Range Supported}
\begin{tabular}{| l | l |} \hline
Tuner	& Frequency range \\ \hline
Elonics E4000	& 52 - 1100 Mhz and 1250 - 2200 MHz 1250 MHz \\ \hline
Rafael Micro R820T &	24 - 1766 MHz \\ \hline
Fitipower FC0013 & 22 - 1100 MHz  \\ \hline
Fitipower FC0012 &	22 - 948.6 MHz \\ \hline
FCI FC2580	& 146 - 308 MHz and 438 - 924 MHz \\ \hline
\end{tabular}
\label{tab:ModelAndFreq}
\end{table}

In the past, people have been able to use the RTL2832U to successfully
receive and decode broadcast FM and air traffic AM radio, TETRA, MGR, GSM, ADS-B
and POCSAG.  This investigation intends to determine its use for signal
identification and classification, a challenging task due to its low signal to
noise ratio, high DC bias, and relatively low sample resolution.  Two
modulation detection methods are proposed, one determining the modulation
based on time domain forth order cumulants.  The second using cyclostationary
statistics found in the frequency domain.  Prototyping of the algorithms
is done in Octave and tested with a controlled simulated input. 
After characterizing the algorithms they will be converted into C++ code
within a GnuRadio for testing with the hardware. 

The long term goal and motivation for this project is to create a REDHAWK C++
component along with a RTL-SDR Device and run a domain consisting of multiple
RTL-SDR Devices so that identification and possibly rough geolocation 
estimation may occur through a network of low cost RTL-SDRs.

