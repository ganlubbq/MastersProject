\chapter{Introduction to Cyclostationary}

Much work has been done on the benefits of using Cyclostationary detection as a
means of energy detection and as a signal classification
technique; particularly in the presence of low SNR \cite{kim2007}.  For an
introduction to Cyclostationary detection \cite{costa1996} contains a lengthy introduction and
examples calculating the theoretical Spectral Correlation Density Function for
an amplitude modulated carrier.

The bulk of the work from this section is a summarization of the work found in
\cite{kim2007}.  A process $x(t)$ is said to be cyclostationary in wide sense
if its mean and auto-correlation are periodic with a period $T_0$.  This may also be expressed as; $M_x(t+T_0) = M_x(t)$ and $R_x(t+T_0, u + T_0) = R_x(t,u)$ \cite{kim2007}. 
We can represent the cyclic auto-correlation function in terms of $t$ and
$\tau$; $R_x(t+\tau/2, t-\tau/2)$ and then, due to its repetitive nature,
rewrite it as a sum of its periodic components in a Fourier Series as shown in
Equation \ref{eq:autoFourier}.

\begin{equation}
R_x(t+\tau/2, t- \tau/2) = \sum_{\alpha} R_x^{\alpha}(\tau)e^{j2\pi \alpha t}
\label{eq:autoFourier}
\end{equation}

Where $\alpha = m/T_0$ and $m$ is an integer. 

The spectral correlation function (SCF) is defined as the Fourier transform of
this cyclic auto-correlation from Equation \ref{eq:autoFourier}.

\begin{equation}
S_x^{\alpha}(f) = \int_{-\infty}^{\infty} R_x^{\alpha}(\tau)e^{-j2\pi f \tau} d\tau
\label{eq:ScdDefined}
\end{equation}

The SCF is similar to the power spectral density of the signal which is defined
as: 

\begin{equation}
S_x(f) = \int_{-\infty}^{\infty} R_x(\tau)e^{-2\pi j f \tau} d\tau = \lim_{T \to \infty} \frac{1}{T}E[|X_T(f)|^2]
\label{eq:PSDDef}
\end{equation}


Do to this relationship, we may also define the SCF in terms of the Fourier
transform of our input signal over the interval $[0, T]$.

\begin{equation}
S_{x_T}^{\alpha}(t,f) \triangleq \frac{1}{T} X_T(t,f + \alpha / 2)X_T^*(t,f - \alpha / 2)
\label{eq:ScdDefined2}
\end{equation}

In practice, there is a trade off here.  We may either increase our time
interval $T$ which is referred to as time smoothing, or smooth over the frequency
domain by using a smaller Fourier transform size.

Like most density functions, the SCD may be scaled to correlation
coefficient that limits its domain to $[0,1]$. This is referred to as the
spectral coherence (SC) and is defined in \ref{eq:specCohDef}.

\begin{equation}
C_{x}^{\alpha}(f) \triangleq \frac{S_x^{\alpha}(f)}{\sqrt{S(f+\alpha/2)S(f-\alpha/2)}}
\label{eq:specCohDef}
\end{equation}


Both the SCD and SC form a 2 dimensional matrix which is often
too large to use as a sensible modulation metric.  Instead, \cite{kim2007}
proposes the concept of the Cycle Frequency Domain Profile (CDP) by taking the
maximum value of the SC over the frequency domain.

\begin{equation}
I(\alpha) \triangleq \max_f |C_x^{\alpha}(f)|
\label{eq:CdpDef}
\end{equation}

The cycle frequency domain profile of known signals can be created and compared
to experimental results to determine the appropriate modulation type.



This section needs more info about how the domain profile differs between each
modulation type and how I plan to use that as a classifier.  Currently, QAM,
BPSK, QAM16 and QAM64 look very similar with 3 spikes while FM, AM and SSB look
somewhat similar with a constant CDP.  Need to investigate more.




